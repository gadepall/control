Find the frequency of oscillation for the Hartley circuit in Fig. \ref{fig:ee18btech11019_fig1}.  Also find the condition on $g_m$.
\begin{figure}[!ht]
	\begin{center}
		\resizebox{\columnwidth}{!}{\input{./figs/ee18btech11019_1.tex}}
	\end{center}
\caption{Hartley oscillator}
\label{fig:ee18btech11019_fig1}
\end{figure}
%Below is the figure, 
\begin{enumerate}[label=\arabic*.,ref=\theenumi]
%\begin{enumerate}[label=\thesection.\arabic*.,ref=\thesection.\theenumi]
\numberwithin{equation}{enumi}
\numberwithin{figure}{enumi}
\item Draw the small signal model for Fig. \ref{fig:ee18btech11019_fig1} and the block diagram for the equivalent control system.
\\
\solution See Figs. \ref{fig:ee18btech11019_fig2} and \ref{fig:ee18btech11019_fig3}.
%
\renewcommand{\thefigure}{\theenumi.\arabic{figure}}
\begin{figure}[!ht]
	\begin{center}
		\resizebox{\columnwidth}{!}{\input{./figs/ee18btech11019_2.tex}}
	\end{center}
\caption{Small signal model}
\label{fig:ee18btech11019_fig2}
\end{figure}

\begin{figure}[!ht]
	\begin{center}
		\resizebox{\columnwidth}{!}{\input{./figs/ee18btech11019_3.tex}}
	\end{center}
\caption{Block diagram}
\label{fig:ee18btech11019_fig3}
\end{figure}
\renewcommand{\thefigure}{\theenumi}

\item Draw the circuit for $H$ and find it.
\\
\solution From Fig. \ref{fig:ee18btech11019_fig7},
\begin{align}
H(s) &= \frac{V_f}{V_o}\\
%V_0 &= V_f + i_1\times \frac{1}{sC}\\
%i_1 &= \frac{V_f}{sL_2}\\
%\end{align}
%Solving,\newline
%\begin{align}
&= \brak{\frac{s^2CL_2}{s^2CL_2 +1}}
\label{eq:ee18btech11019_H}
\end{align}
\begin{figure}[!ht]
	\begin{center}
		\resizebox{\columnwidth}{!}{\input{./figs/ee18btech11019_7.tex}}
	\end{center}
\caption{Small signal H(s)}
\label{fig:ee18btech11019_fig7}
\end{figure}
%
\item Draw the circuit for $G$ and find it.
\\
\solution In Fig. \ref{fig:ee18btech11019_fig8}
%
\begin{align}
    i_1 &= \frac{V_i}{sL_2}
\\
%\end{align}
%\begin{align}
    V_o &= I(sL_1\parallel R)\\
    I &= i_1 + g_mV_i
\end{align}
yielding
\begin{align}
    \frac{V_o}{V_i} = G(s) = \brak{g_m + \frac{1}{sL_2}}\brak{\frac{RsL_1}{R + sL_1}}
\label{eq:ee18btech11019_G}
\end{align}
\begin{figure}[!ht]
	\begin{center}
		\resizebox{\columnwidth}{!}{\input{./figs/ee18btech11019_8.tex}}
	\end{center}
\caption{Small signal G(s)}
\label{fig:ee18btech11019_fig8}
\end{figure}
\item Find the frequency of oscillation.
\\
\solution From \ref{eq:ee18btech11019_G} and \ref{eq:ee18btech11019_G},
%
\begin{align}
    1+G(s)H(s) = 0
\end{align}
\begin{multline}
\implies     s^3(g_mCL_1L_2 + CL_1L_2) +\\ s^2(RCL_1 + RCL_2) + sL_1 + R =0
\end{multline}
For oscillations, substituting $s= \j \omega$, 
%Now, for it to oscillate, roots of the equation should lie on imaginary axis, therefore j$\omega$ should be a solution\newline
%Substituting that, we get
\begin{multline}
    (R - \omega^2(RC(L_1 +L_2)) +\\ j(\omega L_1 - \omega^3(g_mR +1)CL_1L_2) = 0
\end{multline}
Equating the real part of the above to 0,
\begin{align}
    \omega^2(RC(L_1 +L_2) &= R\\
\implies     \omega &= \frac{1}{\sqrt{C(L_1 + L_2)}}
\label{eq:ee18btech11019_f}
\end{align}
Similarly, from the imaginary part, 
\begin{align}
    g_mR + 1 &= \frac{C(L_1 +L_2)}{CL_2}\\
\implies     g_mR &= \frac{L_1}{L_2}
\end{align}
For stable oscillations, 
\begin{align}
g_mR >= \frac{L_1}{L_2}
\end{align}
\numberwithin{figure}{enumi}
\item Simulate the oscillator using  Fig. \ref{fig:ee18btech11019_fig4} and Table \ref{table:ee18btech11019_1}.

\renewcommand{\thefigure}{\theenumi.\arabic{figure}}
\begin{figure}[!ht]
	\begin{center}
		\resizebox{\columnwidth}{!}{\input{./figs/ee18btech11019_4.tex}}
	\end{center}
\caption{Simulation circuit}
\label{fig:ee18btech11019_fig4}
\end{figure}
\begin{table}[!ht]
\centering
\input{./tables/ee18btech11019_1.tex}
\caption{}
\label{table:ee18btech11019_1}
\end{table}
\solution The closed loop impulse response is plotted in Fig.  \ref{fig:ee18btech11019_plot_1}
%\ref{fig:ee18btech11019_fig5}
using the following code.
\begin{lstlisting}
codes/ee18btech11019_1.py
\end{lstlisting}
\begin{figure}[!ht]
\centering
\includegraphics[width=\columnwidth]{./figs/ee18btech11019_5.eps}
\caption{Output when taken from transfer function}
\label{fig:ee18btech11019_plot_1}
\end{figure}
%
The spice simulations are done using the following netlist
\begin{lstlisting}
spice/Draft3.net
\end{lstlisting}
%
and plotted in Fig. \ref{fig:ee18btech11019_plot_2}
%
using the following code.
\begin{lstlisting}
spice/ee18btech11019_2.py
\end{lstlisting}
\begin{figure}[!ht]
\centering
\includegraphics[width=\columnwidth]{./figs/ee18btech11019_6.eps}
\caption{Simulation result}
\label{fig:ee18btech11019_plot_2}
\end{figure}
\renewcommand{\thefigure}{\theenumi}
%
From Fig. \ref{fig:ee18btech11019_plot_2},
the frequency obtained is $3.384$ kHz.  From \ref{eq:ee18btech11019_f}, 
the expected frequency is
\begin{align}
f   &= \frac{1}{2\pi \sqrt{C(L_1 +L_2)}}\\
    &= 3.394 \,kHz
\end{align}
\end{enumerate}
