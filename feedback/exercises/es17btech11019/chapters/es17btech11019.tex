%\begin{enumerate}[label=\thesubsection.\arabic*.,ref=\thesubsection.\theenumi]
\begin{enumerate}[label=\arabic*.,ref=\theenumi]
\numberwithin{equation}{enumi}

\item We are given with a feedback voltage amplifier shown in \ref{fig:Voltage feedback amplifier}.We can neglect $r_{o}$ and given with $R_{1}+R_{2}>>R_{D}$.\\
\begin{figure}[h!]
	\begin{center}
		\resizebox{\columnwidth/1}{!}{\input{./figs/es17btech11019/Voltage_feedback_amplifier.tex}}
	\end{center}
	\caption{}
	\label{fig:Voltage feedback amplifier}
\end{figure}

\item part(a) : We have to find the expressions for $G$(open loop gain) , $H$(the feedback factor) and hence the amount of feedback.\\

\solution
For this , first we have to draw the Small-Signal Model for the above Circuit, we ground all constant voltage sources and open all constant current sources. All Small-Signal paramters are obtained from DC-Analysis of the circuit.In Small-Signal Analysis a N-MOSFET is modelled as a Current Source with value of current equal to $g_{m}v_{gs}$ flowing from Drain to Source. Whereas a P-MOSFET is modelled as a Current Source with value of current equal to $g_{m}v_{sg}$ flowing from Source to Drain.

\begin{figure}[h!]
	\begin{center}
		\resizebox{\columnwidth/1}{!}{\input{./figs/es17btech11019/small_signal_model.tex}}
	\end{center}
	\caption{Small Signal Model}
	\label{fig:Small_Signal}
\end{figure}

%------------------------------------------------------------------------%

\item For finding open loop gain ($G$) and the feedback factor ($H$).\\
\solution 
For finding the open loop gain we have to remove $R_{2}$ and $R_{1}$  and the gate should be grounded.
%------------------------------------------------------------------------%

\item Finding the open loop gain($G$)\\
\solution
\begin{align}
V_{o} = -g_{m}V_{gs}*R_{D}\\
V_{gs} = -V_{S}\\
G = \cfrac{V_{o}}{V_{s}}\\
G = g_{m}R_{D}
\end{align}
%------------------------------------------------------------------------%

\item Finding the Expression for the feedback factor $H$.\\
\solution

\begin{align}
H = \cfrac{V_{f}}{V_{o}}\\
V_{f} = \cfrac{R_{1}}{R_{1}+R_{2}}V_{o}\\
H = \cfrac{R_{1}}{R_{1}+R_{2}}
\end{align}


Amount of feedback is defined as : $1+GH$
\begin{align}
1+GH = 1 + \cfrac{g_{m}R_{D}R_{1}}{R_{1}+R{2}} 
\end{align}

%------------------------------------------------------------------------%



The figure for part(b) :

\begin{figure}[h!]
	\begin{center}
		\resizebox{\columnwidth/1}{!}{
 \begin{circuitikz}[american resistors]
  
  \ctikzset{bipoles/length=1cm}
  
  \draw[color=black]   

    (0,0) to[short](1,0)
    (1.5,0)to[short,*-*,l=$S$](1.5,0)
    
    (0,-2)to[V = $V_{S}$] (0,0)
    
    (0,-2)to[short] node[ground]{}(0,-3)
    
    (4,0)to[cisource, l= $g_{m} V_{gs}$](1,0)
    
    (4,0)to[short,*-*,l=$D$](4,0)
    (4,0)to[short](5,0)
    (5,0)to[short,-o,l=$V_{o}$](6,0)
    
    (5,0) to[R,l_=$R_{D}$,-](5,-3)to[short] node[ground]{}(5,-5)
    
    
    (4,0)to[short](4,-3)
    (4,-2.5)to[short,-o](2,-2.5)
    (2,-2.5)to[short,*-*,l=$G$](2,-2.5)
    node at(2,-1.5){$V_{gs}$}
    node at (2,-0.2){$-$}
    node at (2,-2.2){$+$}
    
    node at(2,-3.5){$V_{f}$}
    node at (2,-2.9){$+$}
    node at (2,-4.7){$-$}
    
    
    (4,-3)  to[short] node[ground]{}(4,-5)
    
    
   
    
    
  ;
 
 
\end{circuitikz}
}
	\end{center}
	\caption{CG amplifier}
	\label{fig:CG amplifier}
\end{figure}

\item Part(b) : We have to eliminate the feedback by removing $R_{1}$ and $R_{2}$ and connecting the gate of Q to a constant DC voltage (signal ground).We have to find the expression of the input resistance  $R_{i}$ and the output resistance $R_{o}$ of the open loop amplifier.\\
\solution\\
When the $R_{1}$ and $R_{2}$ and gate of Q is connected to a constant DC voltage (signal ground) it becomes a CG(Common gate amplifier) without feedback.
We can directly see from the \ref{fig:CG amplifier} the expression of input resistance $R_{i}$ and output resistance $R_{o}$.


For finding input resistance , output constant voltages are grounded and hence the only current flowing is $g_{m}V_{gs}$.Hence $R_{i}$ is :



\begin{align}
I_{in} = -g_{m}V_{gs}\\
V_{in} = V_{S}\\
V_{S} = -V_{gs}\\
R_{i} = \cfrac{V_{in}}{I_{in}}\\
R_{i} = \cfrac{1}{g_{m}}
\end{align}

Similarly , for finding output $R_{o}$ , $V_{in}$ that is $V_{S}$ will be zero and hence $g_{m}V_{gs}$ will be zero. Hence only $R_{D}$ will 
be left which is the output resistance.

\begin{align}
R_{o}=R_{D}
\end{align}
%------------------------------------------------------------------------%
\item Part(c) : Using standard circuit analysis that is without using feedback approach we have to find the input resistance $R_{if}$ and output resistance $R_{of}$ and how they relate to $R_{i}$ and $R_{o}$ , which we find earlier.\\
\solution \\
We will find them one by one.

\begin{figure}[ht!]
	\begin{center}
		\resizebox{\columnwidth}{!}{\input{./figs/es17btech11019/input_resistance.tex}}
	\end{center}
	\caption{}
	\label{fig:Small signal for finding input resistance}
\end{figure}

\item finding expression for $R_{if}$\\
\solution\\

To obtain $R_{if}$ consider the figure \ref{fig:Small signal for finding input resistance} :



We gave test input voltage $V_{x}$ and current $I_{x}$ to find the input resistance from the input side to find $R_{i}$.

\begin{align}
R_{if} = \cfrac{V_{x}}{I_{x}}\\
I_{x} = -g_{m}V_{gs} \\
V_{o} = I_{x}R_{D}\\
V_{f} = \cfrac{V_{o}R_{1}}{R_{1}+R_{2}} = \cfrac{I_{x}R_{D}R_{1}}{R_{1}+R_{2}} \\
V_{x} = -V_{gs} + V_{f} \\
V_{x} = \cfrac{I_{x}}{g_{m}} + \cfrac{I_{x}R_{D}R_{1}}{R_{1}+R_{2}}\\
\cfrac{V_{x}}{I_{x}} =  \cfrac{1}{g_{m}} + \frac{R_{D}R_{1}}{R_{1}+R_{2}}\\
rearranging :\\
R_{if} =  \cfrac{1}{g_{m}}(1+\cfrac{g_{m}R_{D}R_{1}}{R_{1}+R_{2}})\\
R_{if} = R_{i}(1+GH)
\end{align}

The input impedance is increased by a factor of $(1+GH)$.
$R_{if}$ is related to $R_{i}$ by :\\
\begin{align}
R_{if} = R_{i}(1 + GH)
\end{align}

%------------------------------------------------------------------------%



The figure for finding output resistance :

\begin{figure}[ht!]
	\begin{center}
		\resizebox{\columnwidth}{!}{
 \begin{circuitikz}[american resistors]
  
  \ctikzset{bipoles/length=1cm}
  
  \draw[color=black]   

    (0,0)to[short](1,0)
    (1,0)to[short](1,0)
    
    (0,-2)to[short] (0,0)
    
    (0,-2)to[short,l=$S$] node[ground]{}(0,-3)
    
    (4,0)to[cisource, l= $g_{m}V_{gs}$](1,0)
    
    (4,0)to[short,*-*,l=$D$](4,0)
    (4,0)to[short](5,0)
    (6,0)to[short,i=$I_{x}$](5,0)
    (6,-4)to[V=$V_{x}$](6,0)
    (6,-4)to[short] node[ground]{}(6,-5)
    
    (5,0) to[R,l_=$R_{D}$,-](5,-3)to[short] node[ground]{}(5,-5)
    
    (4,0) to[R,l_=$R_{2}$,-](4,-2)
    (4,-2)to[short](4,-3)
    (4,-2.5)to[short,-o](2,-2.5)
    (2,-2.5)to[short,*-*,l=$G$](2,-2.5)
    
    node at(2,-1.5){$V_{gs}$}
    node at (2,-0.2){$-$}
    node at (2,-2.2){$+$}
    
    node at(2,-3.5){$V_{f}$}
    node at (2,-2.9){$+$}
    node at (2,-4.7){$-$}
    
    
    (4,-3) to[R,l_=$R_{1}$,-](4,-4) to[short] node[ground]{}(4,-5)
    
    
   
    
    
  ;
 
 
\end{circuitikz}
}
	\end{center}
	\caption{}
	\label{fig:Small signal for finding output resistance}
\end{figure}

\item finding expression for $R_{of}$\\
\solution\\

To obtain $R_{of}$ consider the figure \ref{fig:Small signal for finding output resistance} :



We gave test input voltage $V_{x}$ and current $I_{x}$ from the output side to find the output resistance and made the input constant voltages as zero.

\begin{align}
R_{of} = \cfrac{V_{x}}{I_{x}}\\
I_{x} = g_{m}V_{gs}(\cfrac{V_{x}}{R_{1}+R_{2}})+(\cfrac{V_{x}}{R_{D}})\\
V_{gs} = \cfrac{R_{1}V_{x}}{R_{1}+R_{2}}\\
I_{x} = \cfrac{g_{m}R_{1}V_{x}}{R_{1}+R_{2}}+\cfrac{V_{x}}{R_{1}+R_{2}}+(\cfrac{V_{x}}{R_{D}})\\
I_{x} = V_{x}(\cfrac{g_{m}R_{1}+1}{R_{1}+R_{2}} + \cfrac{1}{R_{D}})\\
R_{of} = \cfrac{V_{x}}{I_{x}}\\ R_{of} = \cfrac{1}{\cfrac{g_{m}R_{1}+1}{R_{1}+R_{2}}+\cfrac{1}{R_{D}}}
\end{align}



rearranging and multiply both the numerator and denominator by $R_{D}$\\

\begin{align}
R_{of} = \cfrac{R_{D}}{\cfrac{g_{m}R_{1}R_{D}}{R_{1}+R_{2}}+1+\cfrac{R_{D}}{R_{1}+R_{2}}}
\end{align}

since $R_{1}+R_{2}>>R_{D}$ \implies \cfrac{R_{D}}{R_{1}+R_{2}} = 0\\

\begin{align}
R_{of} = \cfrac{R_{D}}{1 + \cfrac{ g_{m}R_{1}R_{D}}{R_{1}+R_{2}}}\\
R_{of} = \cfrac{R_{o}}{1 + GH}
\end{align}

The output impedance is decreased by a factor of $(1+GH)$.
$R_{of}$ is related to $R_{o}$ by :\\
\begin{align}
R_{of} = \cfrac{R_{o}}{1 + GH}
\end{align}

%------------------------------------------------------------------------%

The table showing all the expressions we find out in this problem :

\begin{table}[!ht]
\centering
\input{./tables/es17btech11019/table.tex}
\caption{}
\label{table}
\end{table}


\end{enumerate}
